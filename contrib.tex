2011eric,                      % <20110901eric@outlook.com>
25077667,                      % <zxc25077667@gmail.com>
Arush Sharma,                  % <46960231+arushsharma24@users.noreply.github.com>
asas1asas200,                  % <asas1asas200@gmail.com>
Benno Bielmeier,               % <32938211+bbenno@users.noreply.github.com>
Brad Baker,                    % <brad@brdbkr.com>
ccs100203,                     % <ccs100203@gmail.com>
Chih-Yu Chen,                  % <34228283+chihyu1206@users.noreply.github.com>
ChinYikMing,                   % <yikming2222@gmail.com>
Cyril Brulebois,               % <cyril@debamax.com>
Daniele Paolo Scarpazza,       % <>
David Porter,                  % <>
demonsome,                     % <horseradish1208@gmail.com>
Dimo Velev,                    % <>
Ekang Monyet,                  % <ekangmonyet@posteo.net>
fennecJ,                       % <hwahwa649@gmail.com>
Francois Audeon,               % <>
gagachang,                     % <vivahavey@gmail.com>
Gilad Reti,                    % <gilad.reti@gmail.com>
Horst Schirmeier,              % <>
Hsin-Hsiang Peng,              % <hsinspeng@gmail.com>
Ignacio Martin,                % <>
JianXing Wu,                   % <fdgkhdkgh@gmail.com>
linD026,                       % <shiyn.lin@gmail.com>
Marconi Jiang,                 % <marconi1964@yahoo.com>
RinHizakura,                   % <s921975628@gmail.com>
Roman Lakeev,                  % <>
Stacy Prowell,                 % <sprowell@gmail.com>
Tucker Polomik,                % <tucker.polomik@inficon.com>
VxTeemo,                       % <tcccvvv123@gmail.com>
Wei-Lun Tsai,                  % <alan23273850@gmail.com>
xatier,                        % <xatierlike@gmail.com>
Ylowy.                         % <69316865+YLowy@users.noreply.github.com>
